\problemname{FizzBuzz}
Ett företag ska anställa en mjukvaruutvecklare till ett topphemligt jobb.
Efter mycket headhunting har de nu endast $N$ stycken kandidater kvar, efter anonymisering numrerade från $1$ till $N$.
För att testa kandidaternas förmåga att programmera bestämmer sig företaget för att be dem implementera FizzBuzz.

En korrekt implementation av FizzBuzz gör följande.
För vart och ett av talen $1$ till $M$,
\begin{itemize}
 \item om talet är delbart med 3 skrivs ``fizz'' ut,
 \item om talet är delbart med 5 skrivs ``buzz'' ut,
 \item om talet är delbart med både 3 och 5 skrivs ``fizzbuzz'' ut, och
 \item om talet varken är delbart med 3 eller 5 skrivs talet självt ut. 
\end{itemize}

Då jobbet är topphemligt är företaget mycket noga med att exakt rätt sak skrivs ut på rätt position.
Din uppgift är att hitta den kandidaten vars program skrev ut rätt sak på rätt position flest gånger.

\section*{Indata}
Den första raden av indatan består av heltalen $N$ ($2 \leq N \leq 1000$) och $M$ ($2 \leq M \leq 100$), antalet kandidater och antalet värden deras program ska skriva ut.

Därefter följer $N$ rader med $M$ strängar bestående av endast små bokstäver \texttt{a-z} och siffror $0$-$9$, separerade med blanksteg.
Rad $i$ innehåller utdatan från kandidat $i$.
Samtliga $M$ strängar i utskriften från en kandidats program innehåller minst ett tecken.

\section*{Utdata}
Skriv ut numret på den kandidat vars utskrift har rätt ord på rätt plats flest antal gånger.
Om flera kandidater har flest rätt, skriv ut den med lägst nummer.
