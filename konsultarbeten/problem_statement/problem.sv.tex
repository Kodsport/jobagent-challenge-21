\problemname{Jobbyte}
I den datorgenererade världen Matriks jobbar du med att designa unika rutchkanor till coola kontor. Därför jobbar du förstås som konsult. 
Företag brukar vilja matcha sina rutchkanor med den senaste tekniken de har lanserat och alla coola företag lanserar ny teknik samma vecka varje år. T.ex har lanserar alltid företaget Päron nya walkie-talkies första veckan i oktober och företaget Vesla lanserar alltid nya elcyklar i mitten på april.
När du på nyårsafton tittar igenom din mailkorg hittar du därför $N$ stående jobberbjudanden med start olika veckor på året. Du vet att du alltid själv kan välja att jobba för ett företag 13, 26 eller 39 veckor (1, 2 eller 3 kvartal) och att du aldrig kommer få några nya jobberbjudanden i din karriär. 
Hur ska du planera de $M$ åren som är kvar av din karriär så att du är arbetslös så få veckor som möjligt?

(Det är alltid exakt 52 veckor på ett år och du kan tänka dig jobba lite efter att din karriär är slut, bara inte ta några nya jobb.)

\section*{Indata}
Två tal $1 \leq N \leq 52$ och $1 \leq M \leq 10^9$, antalet jobberbjudanden och antalet år kvar på din karriär. 
En rad med $N$ st tal. Tal $a_i$ är veckan som företag $i$ vill att du ska börja jobba. Inga två företag vill att du ska börja arbeta samma vecka.

\section*{Utdata}
Ett heltal, det minsta antalet veckor du måste vara arbetslös. 
